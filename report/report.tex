\documentclass[12pt,english]{article}
\usepackage[a4paper,bindingoffset=0.2in,%
left=1in,right=1in,top=1in,bottom=1in,%
footskip=.25in]{geometry}

\usepackage[backend=bibtex,citestyle=ieee,alldates=iso8601]{biblatex}
\usepackage{hyperref}
\addbibresource{report.bib}

\title{Pendulum with Friction}
\author{Raj Krishnan}

\begin{document}

\maketitle

\section{Introduction}
  The primary forces acting on the pendulum bob are the gravitational force 
  that makes it move in the first place and the force exerted by the string
  to keep it moving along a circular path \cite{surrow}. In addition, there may be a 
  damping force from friction at the pivot or air resistance or both. We 
  will model how the angle made by the pendulum with it's mean position and
  the angular velocity change with time.

\section{Equations of Motion}
  
  The forces acting on the pendulum are gravitation, tension and a damping
  force. A damping force, modeling for example the viscous damping of the 
  oil in the bearing at the pendulum hinge, can as an approximation be 
  proportional to the angular velocity of the pendulum, with a coefficient 
  $\alpha$. We can write the equations \cite{james} in order to obtain the 
  following:
  \begin{equation}
          \label{eq:eom}
          \ddot{\theta} = -\frac{g}{L}sin(\theta) - \alpha \dot{\theta}                    
  \end{equation}

 \section{Methodology Followed}

   The code consists of a \texttt{Oscillator} class, which is initialized with 
   the initial state, the length of the pendulum rod/wire and the damping factor
   $\alpha$. 

   We now use \texttt{odeint} in \texttt{scipy.integrate} to solve the equation 
   for each step after computing the derivatives as follows:
   
   $$\frac{dy}{dx} = \dot{\theta}$$
   $$\frac{d^2y}{dx^2} = \ddot{\theta} 
                       = -\frac{g}{L}sin(\theta) - \alpha \dot{\theta}$$

   The trajectory of the pendulum bob is now calculated and we save it's angle
   and angular velocity.

\printbibliography

\end{document}
