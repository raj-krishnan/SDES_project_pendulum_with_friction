\documentclass[12pt,english]{article}
\usepackage[a4paper,bindingoffset=0.2in,%
left=1in,right=1in,top=1in,bottom=1in,%
footskip=.25in]{geometry}
% Raj Krishnan 140010007
\usepackage[backend=bibtex,citestyle=ieee,alldates=iso8601]{biblatex}
\usepackage{hyperref}
\usepackage{graphicx}
\usepackage{amsmath}
\usepackage{amsfonts,amssymb,amsthm}
\usepackage{cleveref}
\addbibresource{report.bib}

\title{Pendulum with Friction}
\author{Raj Krishnan}

\begin{document}

\maketitle

\textbf{The source code for this assignment is available at } \url{https://github.com/raj-krishnan/SDES_project_pendulum_with_friction}

\section{Introduction}
  The primary forces acting on the pendulum bob are the gravitational force 
  that makes it move in the first place and the force exerted by the string
  to keep it moving along a circular path \cite{surrow}. In addition, there may be a 
  damping force from friction at the pivot or air resistance or both. We 
  will model how the angle made by the pendulum with it's mean position and
  the angular velocity change with time.

\section{Equations of Motion}
  
  The forces acting on the pendulum are gravitation, tension and a damping
  force. A damping force, modeling for example the viscous damping of the 
  oil in the bearing at the pendulum hinge, can as an approximation be 
  proportional to the angular velocity of the pendulum, with a coefficient 
  $\alpha$. We can write the equations \cite{james} in order to obtain the 
  following:
  \begin{equation}
          \label{eq:eom}
          \ddot{\theta} = -\frac{g}{L}sin(\theta) - \alpha \dot{\theta}                    
  \end{equation}

\section{Methodology Followed}

  The code consists of a \texttt{Oscillator} class, which is initialized with 
  the initial state, the length of the pendulum rod/wire and the damping factor
  $\alpha$. 

  We now use \texttt{odeint} in \texttt{scipy.integrate} to solve the equation 
  for each step after computing the derivatives as follows:
   
  $$\frac{dy}{dx} = \dot{\theta}$$
  $$\frac{d^2y}{dx^2} = \ddot{\theta} 
                      = -\frac{g}{L}sin(\theta) - \alpha \dot{\theta}$$

  The trajectory of the pendulum bob is now calculated and we save it's angle
  and angular velocity.

 \section{Dependencies}
   The dependencies that are required to be installed are:

   \begin{itemize}
     \item Python 3
     \item Latex
     \item Bibtex
     \item numpy, scipy, matplotlib
   \end{itemize}

\section{Plots and Animations}
  The oscillator class is now instantiated to obtain plots under various
  scenarios. \cref{fig:plot_pendulum} contains a plot of angular displacement with time for 
  various damping conditions
  \begin{figure}[h!]
    \centering
    \includegraphics[width=0.8\textwidth]{build/pendulum.png}
    \caption{Plot showing the change in angular displacement with time for various damping factors}
    \label{fig:plot_pendulum}
  \end{figure}

  
  \cref{fig:plot_theta_relation} shows the relationship between the angular velocity of the body and its
  angular displacement, and we can see how energy is lost due to damping.
  \begin{figure}[h!]
    \centering
    \includegraphics[width=0.8\textwidth]{build/theta_vs_theta_dash.png}
    \caption{Plot showing the angular displacement against the angular velocity}
    \label{fig:plot_theta_relation}
  \end{figure}

  The animations for the assignment are available in the html file 140010007.html. 
  The animations include the evolution of angular displacement and velocity 
  over time, as well as a model of a damped simple pendulum. 
  
 
   % The animations for this assignment can be found \url{file://../output/report.pdf}{here}

\printbibliography

\end{document}
